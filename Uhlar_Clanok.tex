% Metódy inžinierskej práce

\documentclass[10pt,twoside,slovak,a4paper]{article}

\usepackage[slovak]{babel}
%\usepackage[T1]{fontenc}
\usepackage[IL2]{fontenc} % lepšia sadzba písmena Ľ než v T1
\usepackage[utf8]{inputenc}
\usepackage{graphicx}
\usepackage{url} % príkaz \url na formátovanie URL
\usepackage{hyperref} % odkazy v texte budú aktívne (pri niektorých triedach dokumentov spôsobuje posun textu)

\usepackage{cite}
%\usepackage{times}

\pagestyle{headings}

\title{Modelovanie spracovania elektronického podania ÚPVS sekvenčnými UML diagramami \thanks{Semestrálny projekt v predmete Metódy inžinierskej práce, ak. rok 2021/22, vedenie: Vladimír Mlynarovič}} % meno a priezvisko vyučujúceho na cvičeniach

\author{Viktor Uhlár\\[2pt]
	{\small Slovenská technická univerzita v Bratislave}\\
	{\small Fakulta informatiky a informačných technológií}\\
	{\small \texttt{xuhlar@stuba.sk}}
	}

\date{\small 6. november 2021} % upravte



\begin{document}

\maketitle

\begin{abstract}


{Ústredný portál verejnej správy (ÚPVS) je centrálnym miestom na podávanie a spracovanie elektronických podaní. Pre agendové informačné systémy poskytuje možnosť integrácie tak, aby bolo možné zasielať elektronické podania priamo z vlastných informačných systémov (teda bez nutosti priuhlasovania sa do elektronických schránok pomocou eID - elektronického občianskeho preukazu).}\\
{Keďže postupnosť spracovania elektronického podania závisí od jeho účelu, musí byť pred fázou samotnej integrácie vypracovaný model "workflow". ÚPVS pre tento účel vyžaduje tzv. Dohodu io integračnom zámere, kde je spracovanie modelované vo forme sekvenčného UML diagramu.
V tejto práci sa teda budem venovať teórii modelovania sekvenčných UML modelov s konrétny príkladom pre spracovanie elektronického rozhodnutia "fiktívneho"  OVM (orgánu verejnej moci SR).}

\end{abstract}



\section{Úvod}

Ústredný portál verejnej správy (ÚPVS) je centrálnym miestom na podávanie a spracovanie elektronických podaní. Pre agendové informačné systémy poskytuje možnosť integrácie tak, aby bolo možné pristupovať k elektronickej schránke priamo z vlastných informačných systémov - teda bez nutosti prihlasovania sa eID - elektronického občianskeho preukazu.\\

Keďže spôsob pripojenia môže byť rôzny, správca ÚPVS (Národná agentúra pre sieťové a elektronické služby - NASES) vyžaduje podrobný integračný zámer, ktorého súčasťou je aj modelovanie integrácie vo fornme sekvenčného UML diagramu.\\ 

V tejto práci sa teda budem venovať:\\

\begin{itemize}
\item Rámcovému popisu UML a typov diagramov
\item Spôsobu tvorby sekvenčných UML diagramov
\item Jednoduchému nástroju pre modelovanie UML diagramov
\item Modelovej integračnej situácie pripojenia na ÚPVS a popisu tvorby súvisiaceho sekvenčného UML diagramu
\end{itemize}


Základný problém, ktorý bol naznačený v úvode, je podrobnejšie vysvetlený v časti~\ref{nejaka}.
Dôležité súvislosti sú uvedené v častiach~\ref{dolezita} a~\ref{dolezitejsia}.
Záverečné poznámky prináša časť~\ref{zaver}.



\section{Popis UML a rozdelenie typov diagramov} \label{Popis}

Z obr.~\ref{f:rozhod} je všetko jasné. 

\begin{figure*}[tbh]
\centering
%\includegraphics[scale=1.0]{diagram.pdf}
Aj text môže byť prezentovaný ako obrázok. Stane sa z neho označný plávajúci objekt. Po vytvorení diagramu zrušte znak \texttt{\%} pred príkazom \verb|\includegraphics| označte tento riadok ako komentár (tiež pomocou znaku \texttt{\%}).
\caption{Rozhodujúci argument.}
\label{f:rozhod}
\end{figure*}



\section{Sekvenčný diagram a jeho modelovanie} \label{Sekvenčný}

Základným problémom je teda\ldots{} Najprv sa pozrieme na nejaké vysvetlenie (časť~\ref{ina:nejake}), a potom na ešte nejaké (časť~\ref{ina:nejake}).\footnote{Niekedy môžete potrebovať aj poznámku pod čiarou.}

Môže sa zdať, že problém vlastne nejestvuje\cite{Coplien:MPD}, ale bolo dokázané, že to tak nie je~\cite{Czarnecki:Staged, Czarnecki:Progress}. Napriek tomu, aj dnes na webe narazíme na všelijaké pochybné názory\cite{PLP-Framework}. Dôležité veci možno \emph{zdôrazniť kurzívou}.


\section{Nástroj na modelovanie formou sekvenčného UML diagramu} \label{Nástroj}

Niekedy treba uviesť zoznam:

\begin{itemize}
\item jedna vec
\item druhá vec
	\begin{itemize}
	\item x
	\item y
	\end{itemize}
\end{itemize}

Ten istý zoznam, len číslovaný:

\begin{enumerate}
\item jedna vec
\item druhá vec
	\begin{enumerate}
	\item x
	\item y
	\end{enumerate}
\end{enumerate}


\section{Integračný scenár a jeho modelovanie} \label{Integračný}

\paragraph{Veľmi dôležitá poznámka.}
Niekedy je potrebné nadpisom označiť odsek. Text pokračuje hneď za nadpisom.



\section{Niečo mozno ešte} \label{Niečo}





\section{Záver} \label{zaver} % prípadne iný variant názvu



%\acknowledgement{Ak niekomu chcete poďakovať\ldots}


% týmto sa generuje zoznam literatúry z obsahu súboru literatura.bib podľa toho, na čo sa v článku odkazujete
\bibliography{Uhlar_Literatura}
\bibliographystyle{abbrv} % prípadne alpha, abbrv alebo hociktorý iný
\end{document}
